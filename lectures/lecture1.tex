% Lecture 1: Course Introduction & State Estimation Overview
\documentclass[aspectratio=169]{beamer}
\usetheme{Madrid}
\usecolortheme{whale}

% Packages
\usepackage{amsmath}
\usepackage{mathtools}
\usepackage{graphicx}
\usepackage{hyperref}
\usepackage{xcolor}
\usepackage{listings}

% Define colors for syntax highlighting
\definecolor{codegreen}{rgb}{0,0.6,0}
\definecolor{codegray}{rgb}{0.5,0.5,0.5}
\definecolor{codepurple}{rgb}{0.58,0,0.82}
\definecolor{backcolour}{rgb}{0.95,0.95,0.92}

% Configure Python style
\lstdefinestyle{pythonstyle}{
    language=Python,
    basicstyle=\ttfamily\tiny,
    backgroundcolor=\color{backcolour},
    commentstyle=\color{codegreen},
    keywordstyle=\color{blue}\bfseries,
    stringstyle=\color{codepurple},
    breakatwhitespace=false,
    breaklines=true,
    captionpos=b,
    keepspaces=true,
    numbers=left,
    numberstyle=\tiny\color{codegray},
    numbersep=5pt,
    showspaces=false,
    showstringspaces=false,
    showtabs=false,
    tabsize=4,
    frame=single,
    morekeywords={self,def,class,return,import,from,as,with},
    emphstyle={\color{blue}},
    emph={numpy,scipy,np,minimize,norm,multivariate_normal}
}

\lstset{style=pythonstyle}

% Define argmin/argmax operators
\DeclareMathOperator*{\argmin}{arg\,min}
\DeclareMathOperator*{\argmax}{arg\,max}

% Title Page Info
\title{SES/RAS 598: Space Robotics and AI}
\subtitle{Lecture 1: Course Introduction \& State Estimation Overview}
\author{Dr. Jnaneshwar Das}
\institute{Arizona State University \\ School of Earth and Space Exploration}
\date{Spring 2025}

\begin{document}

% Title slide
\begin{frame}
    \titlepage
\end{frame}

% Outline
\begin{frame}{Lecture Outline}
    \tableofcontents
\end{frame}

\section{Course Overview}

\begin{frame}{Course Structure}
    \begin{itemize}
        \item<1-> \textbf{Meeting Times:} Tu/Th 10:30-11:45am
        \item<2-> \textbf{Location:} PSF 647
        \item<3-> \textbf{Course Components:}
            \begin{itemize}
                \item Assignments (20\%)
                \item Midterm Project (20\%)
                \item Final Project (50\%)
                \item Class Participation (10\%)
            \end{itemize}
        \item<4-> \textbf{Prerequisites:}
            \begin{itemize}
                \item Linear algebra, calculus, probability theory
                \item Python programming with NumPy, SciPy
                \item Basic computer vision concepts
                \item Linux/Unix systems experience
            \end{itemize}
    \end{itemize}
\end{frame}

\begin{frame}{Course Resources}
    \begin{itemize}
        \item<1-> \textbf{Recommended Books:}
            \begin{itemize}
                \item Probabilistic Robotics (Thrun, Burgard, Fox)
                \item Optimal State Estimation (Simon)
                \item Pattern Recognition and Machine Learning (Bishop)
            \end{itemize}
        \item<2-> \textbf{Interactive Tutorials:}
            \begin{itemize}
                \item Sensor Fusion
                \item Parameter Estimation
                \item Gaussian Processes
            \end{itemize}
        \item<3-> \textbf{Required Software:}
            \begin{itemize}
                \item Linux OS
                \item ROS2
                \item Python with scientific computing libraries
            \end{itemize}
    \end{itemize}
\end{frame}

\section{State Estimation Fundamentals}

\begin{frame}{Why State Estimation?}
    \begin{itemize}
        \item<1-> \textbf{Real-World Applications:}
            \begin{itemize}
                \item Mars rover navigation
                \item Drone flight control
                \item Satellite attitude determination
            \end{itemize}
        \item<2-> \textbf{Key Challenges:}
            \begin{itemize}
                \item Sensor noise and uncertainty
                \item Environmental dynamics
                \item Resource constraints
            \end{itemize}
        \item<3-> \textbf{Impact on Space Exploration:}
            \begin{itemize}
                \item Autonomous navigation
                \item Precision landing
                \item Sample collection
            \end{itemize}
    \end{itemize}
\end{frame}

\begin{frame}{Least Squares Estimation}
    \begin{itemize}
        \item<1-> \textbf{Mathematical Foundation:}
            \[ \hat{\theta} = \argmin_{\theta} \sum_{i=1}^n (y_i - h(\theta))^2 \]
        \item<2-> \textbf{Key Properties:}
            \begin{itemize}
                \item Minimizes squared error
                \item Optimal for Gaussian noise
                \item Computationally efficient
            \end{itemize}
        \item<3-> \textbf{Applications:}
            \begin{itemize}
                \item Sensor calibration
                \item Trajectory estimation
                \item Parameter identification
            \end{itemize}
    \end{itemize}
\end{frame}

\begin{frame}[fragile]{Implementation Example: Least Squares Estimation}
\begin{lstlisting}[language=Python]
import numpy as np
from scipy.optimize import minimize

class LeastSquaresEstimator:
    def __init__(self, measurements, measurement_model):
        self.y = measurements        # Measurement vector
        self.h = measurement_model   # Measurement model function
        
    def cost_function(self, theta):
        """Compute sum of squared errors."""
        residuals = self.y - self.h(theta)
        return np.sum(residuals**2)
    
    def estimate(self, theta_init):
        """Find parameters that minimize squared error."""
        result = minimize(self.cost_function, theta_init, 
                        method='Nelder-Mead')
        return result.x  # Return optimal parameters
\end{lstlisting}
\end{frame}

\begin{frame}{Maximum Likelihood Estimation}
    \begin{itemize}
        \item<1-> \textbf{Principle:}
            \[ \hat{\theta}_{\text{MLE}} = \argmax_{\theta} \prod_{i=1}^n p(y_i|\theta) \]
        \item<2-> \textbf{Connection to Least Squares:}
            \begin{itemize}
                \item Equivalent under Gaussian assumptions
                \item More general framework
                \item Handles different noise models
            \end{itemize}
        \item<3-> \textbf{Space Applications:}
            \begin{itemize}
                \item Orbit determination
                \item Attitude estimation
                \item Sensor fusion
            \end{itemize}
    \end{itemize}
\end{frame}

\begin{frame}[fragile]{Implementation Example: Maximum Likelihood Estimation}
\begin{lstlisting}[language=Python]
import numpy as np
from scipy.stats import norm
from scipy.optimize import minimize

class MLEstimator:
    def __init__(self, measurements, measurement_model):
        self.y = measurements        # Measurement vector
        self.h = measurement_model   # Measurement model function
        
    def neg_log_likelihood(self, theta):
        """Compute negative log-likelihood."""
        residuals = self.y - self.h(theta)  # Assuming Gaussian noise model
        return -np.sum(norm.logpdf(residuals))
    
    def estimate(self, theta_init):
        """Find parameters that maximize likelihood."""
        result = minimize(self.neg_log_likelihood, theta_init, 
                        method='Nelder-Mead')
        return result.x  # Return optimal parameters
\end{lstlisting}
\end{frame}

\section{Linear Dynamical Systems}

\begin{frame}{State-Space Models}
    \begin{itemize}
        \item<1-> \textbf{System Dynamics:}
            \begin{align*}
                x_{k+1} &= Ax_k + Bu_k + w_k \\
                y_k &= Cx_k + v_k
            \end{align*}
        \item<2-> \textbf{Components:}
            \begin{itemize}
                \item State vector $x_k$
                \item Input vector $u_k$
                \item Measurement vector $y_k$
                \item Process noise $w_k$
                \item Measurement noise $v_k$
            \end{itemize}
    \end{itemize}
\end{frame}

\begin{frame}{Case Study: Mars Rover Navigation}
    \begin{itemize}
        \item<1-> \textbf{State Variables:}
            \begin{itemize}
                \item Position (x, y, z)
                \item Orientation (roll, pitch, yaw)
                \item Velocities
            \end{itemize}
        \item<2-> \textbf{Sensors:}
            \begin{itemize}
                \item Visual odometry
                \item Inertial measurement unit (IMU)
                \item Sun sensors
            \end{itemize}
        \item<3-> \textbf{Challenges:}
            \begin{itemize}
                \item Wheel slippage
                \item Varying terrain
                \item Limited computational resources
            \end{itemize}
    \end{itemize}
\end{frame}

\begin{frame}[fragile]{Implementation Example: State-Space Model}
\begin{lstlisting}[language=Python]
import numpy as np
from scipy.stats import multivariate_normal

class LinearStateSpaceModel:
    def __init__(self, A, B, C, Q, R):
        self.A = A  # State transition matrix
        self.B = B  # Input matrix
        self.C = C  # Measurement matrix
        self.Q = Q  # Process noise covariance
        self.R = R  # Measurement noise covariance
        
    def propagate_state(self, x, u=None):
        """Propagate state forward one step."""
        w = multivariate_normal.rvs(mean=np.zeros(x.shape), cov=self.Q)
        if u is not None:
            return self.A @ x + self.B @ u + w
        return self.A @ x + w
    
    def get_measurement(self, x):
        """Get noisy measurement of current state."""
        v = multivariate_normal.rvs(mean=np.zeros(self.C.shape[0]), cov=self.R)
        return self.C @ x + v
\end{lstlisting}
\end{frame}

\section{Next Steps}

\begin{frame}{Preparation for Next Lecture}
    \begin{itemize}
        \item<1-> \textbf{Review:}
            \begin{itemize}
                \item Matrix operations
                \item Probability concepts
                \item Basic Python programming
            \end{itemize}
        \item<2-> \textbf{Setup:}
            \begin{itemize}
                \item Install Linux if needed
                \item Configure ROS2 environment
                \item Test Python scientific libraries
            \end{itemize}
        \item<3-> \textbf{Reading:}
            \begin{itemize}
                \item Skim Kalman filter basics
                \item Review assigned papers
                \item Explore interactive tutorials
            \end{itemize}
    \end{itemize}
\end{frame}

\begin{frame}{Questions?}
    \begin{center}
        \Huge Thank you!
        
        \vspace{1cm}
        \normalsize
        Contact: jdas5@asu.edu
    \end{center}
\end{frame}

\begin{frame}{Course Assessment: Quiz Components}
\begin{itemize}
    \item \textbf{Part 1: Robotics and AI Fundamentals}
    \begin{itemize}
        \item Tests foundational knowledge in robotics and AI
        \item Key topics: SLAM, LiDAR, occupancy grids, GPS challenges, path planning
        \item Required score: 60\% to proceed to Part 2
    \end{itemize}
    \item \textbf{Part 2: Advanced Concepts}
    \begin{itemize}
        \item Tests understanding of tutorial concepts
        \item Topics covered:
        \begin{itemize}
            \item Multi-view geometry and stereo vision
            \item SLAM systems and loop closure
            \item Kalman filter parameters and tuning
            \item Sampling strategies for exploration
            \item Error analysis in rock mapping
        \end{itemize}
    \end{itemize}
\end{itemize}
\end{frame}

\begin{frame}{Quiz Topics and Course Content}
\begin{itemize}
    \item \textbf{Foundational Topics (Part 1)}
    \begin{itemize}
        \item SLAM and LiDAR: Core to robot perception and mapping
        \item Occupancy grids: Probabilistic environment representation
        \item GPS challenges: Motivation for robust state estimation
        \item Path planning: Essential for autonomous navigation
    \end{itemize}
    \item \textbf{Advanced Applications (Part 2)}
    \begin{itemize}
        \item Stereo vision: Depth estimation for rock mapping
        \item Loop closure: Improving map consistency
        \item Kalman filters: State estimation for robot localization
        \item Sampling strategies: Efficient exploration algorithms
        \item Error analysis: Ensuring reliable measurements
    \end{itemize}
\end{itemize}
\end{frame}

\end{document} 